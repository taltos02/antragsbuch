\sonstigerantrag{X06}{Altanträge als Positionspapiere}
{http://wiki.piratenpartei.de/BY:Landesparteitag 2012.1/Antragsfabrik/Altanträge_als_Positionspapiere}
{CEdge}
\subsection{Antrag}
Der Landesparteitag möge folgendes beschließen:\\
Die auf dem Landesparteitag 2010.1 beschlossenen Anträge \glqq Kennzeichnung von Polizeibeamten\grqq\ ,
\glqq Software in der öffentlichen Verwaltung\grqq\  und \glqq \glqq Neue\grqq\  Grundrechte\grqq\  werden, soweit nicht durch
andere Beschlüsse ersetzt, zu Positionspapieren umdefiniert.

\subsection{Begründung}
Diese Anträge entstanden vor der dem Programmentwicklungskonzept und waren für das
Landeswahlprogramm gedacht. Sie eignen sich ohne weiteres als Positionspapiere, da sie sehr
ausführlich sind und auch eine Begründung enthalten.\par
Durch den Beschluss als Positionspapier können die Inhalte der Beschlüsse ins Wahlprogramm
einfließen.\par
Die betroffenen Anträge stammen ausschließlich vom Urheber dieses Antrags. Dieser Antrag dürfte
eine 2/3-Mehrheit benötigen, da die umdeklarierten Programmanträge mit selbiger beschlossen
wurden.\par
Siehe auch\\
\url{http://wiki.piratenpartei.de/Landesverband_Bayern/Landeswahlprogramm}\\
\url{http://wiki.piratenpartei.de/Archiv:Antragsfabrik_Bayern/Identifikation_von_Polizeikräften}\\
\url{http://wiki.piratenpartei.de/Archiv:Antragsfabrik_Bayern/Software_in_der_öffentlichen_Verwaltung}\\
\url{http://wiki.piratenpartei.de/Archiv:Antragsfabrik_Bayern/"Neue"_Grundrechte}

\sonstigerantrag{X07}{Redaktionskommission}{http://wiki.piratenpartei.de/BY:Landesparteitag 2012.1/Antragsfabrik/Redaktionskommission}{TurBor}
\subsection{Antrag}
Der Landesparteitag möge sich dafür aussprechen, dass\\
bis zum nächsten Landesparteitag eine redaktionelle Bearbeitung des Parteiprogramms
durchgeführt wird. Dabei soll das Programm und die beschlossenen Positionspapiere - insbesondere
die auf dem Landesparteitag 2012.1 neu beschlossenen Punkte - klar und logisch strukturiert, von
sprachlichen Mängeln bereinigt und stilistisch einheitlich gestaltet werden. Es dürfen keine
inhaltlichen Veränderungen vorgenommen werden.\par
Die Überarbeitung wird durch eine durch den Vorstand oder den Landesparteitag einberufene
Programmkommission in Zusammenarbeit mit den Autoren der betroffenen Programmpunkte
durchgeführt.\par
Das so überarbeitete Programm bzw. Positionspapiere müssen, um Gültigkeit zu erlangen, durch
den nächsten Landesparteitag ratifiziert werden. Zu diesem Zweck möge der Landesvorstand die
vorgeschlagene Überarbeitung fristgerecht vor dem nächsten Landesparteitag zur parteiinternen
Diskussion stellen und einreichen.

\subsection{Begründung}
Sollte sogar ein geringer Anteil der vorgeschlagenen Programmänderungsanträge angenommen
werden, wird unser Parteiprogramm ziemlich chaotisch aussehen, und da die Anträge von sehr
vielen verschiedenen Personen stammen, ist weder eine einheitliche Struktur noch ein einheitlicher
Stil gewährleistet. Hinzu kommt, dass manche Anträge trotz inhaltlicher Stärke sprachlich schwach
sind.\par
Der Antrag zielt darauf ab, bereits auf dem LPT einen Übrarbeitungsvorgang einzuleiten, damit wir
auf dem nächsten Parteitag die überarbeitete Version ratifizieren können. Es handelt sich dabei nicht
um ein reines Korrekturlesen (Rechtschreib-/Grammatikfehler), da auch die Struktur sowie die
stilistischen Gegebenheiten geändert werden sollten. Der Inhalt muss natürlich in vollem Umfang
erhalten bleiben. Die eigentliche Arbeit wird wahrscheinlich von beauftragten Piraten durchgeführt,
der Vorstand ist aber für die Umsetzung verantwortlich.\par
Falls der Antrag angenommen wird, kann auch die Diskussion über alle nachfolgenden Anträge auf
dem Parteitag sich auf deren Inhalt und nicht eventuelle sprachliche Schwächen konzentrieren.
