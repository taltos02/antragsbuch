\anderungsantrag
% Antragsnummer
{S01}
% Antragstitel
{Landesprogramm des LV Bayern}
% Wikiurl zum Antrag
{http://wiki.piratenpartei.de/BY:Landesparteitag 2012.1/Antragsfabrik/Landesprogramm des LV Bayern}
% Antragssteller
{Fard}
% Betrifft
{Satzung des Landesverbands Bayern / §11(3)}
\konkurrenz{S02}
\subsection{Beantragte Änderung}
§11(3) der Satzung des Landesverbands Bayern wird wie folgt neugefasst:\\
1 Der Landesverband übernimmt das Grundsatzprogramm der Piratenpartei Deutschland. 2 Vom
Landesparteitag kann ein eigenes Landesprogramm für den Landesverband Bayern sowie
Wahlprogramme für Landtagswahlen verabschiedet werden. 3 Diese müssen auf den Werten des
Grundsatzprogrammes basieren.
\subsection{Begründung}
\subsubsection{Zuerst eine Übersicht der Änderungen:}
\begin{quotation}
1Der Landesverband übernimmt das Grundsatzprogramm der Piratenpartei Deutschland. 2Vom
Landesparteitag kann ein eigenes Wahlprogramm für Kommunal- und Landtagswahlen
verabschiedet werden. 3Dieses muss auf den Werten des Grundsatzprogrammes basieren.
\end{quotation}
Im 2. Satz wird der Verweis auf Programme für Kommunalwahlen gestrichen (ist nicht Aufgabe des
LVs/LPTs). Ausserdem wird die Möglichkeit geschaffen ein eigenes Landesprogramm zu erstellen.
Dies war bisher nicht vorgesehen.
\subsubsection{Nun zum eigentlichen Sinn des Antrags:}
Die sehr schleppend vorangehende Programmarbeit im LV ist zum Teil darauf zurückzuführen, dass
die Landesebene garnicht als \glqq Programmebne\grqq\  wahrgenommen wird. Auf Landesebene war bisher
das beste mögliche Schicksal für einen Programmantrag dass er in den Untiefen des Wikis als
\glqq Positionspapier\grqq\  Staub sammelt. Speziell für landespolitische Themen (Bildung, Polizei, etc.)
macht es Sinn neben konkreten Inhalten i

\subsubsection{Unterschied zum Antrag}
\url{http://wiki.piratenpartei.de/BY:Landesparteitag_2012.1/Antragsfabrik/Grundsatzprogramm_des_LV_Bayern}:\\
Im Gegensatz zum Antrag \glqq Grundsatzprogramm des LV Bayern\grqq\  bleibt bei diesem Antrag der
erste Satz {\quote Der Landesverband übernimmt das Grundsatzprogramm der Piratenpartei
Deutschland.}  in der Satzung. Falls der Antrag von Boris angenommen wird, möchte dieser das
aktuelle Bundesprogramm einzeln auf dem LPT abstimmen lassen. Dabei würde ein
Landesgrundsatzprogramm entstehen, welches nur noch die im LV Bayern mehrheitsfähigen
Positionen enthält. Dieses Landesgrundsatzprogramm wäre somit eine alternative bayerische
Version des Bundesprogrammes. Dies ist meiner Meinung nach ein erster Schritt hin zu einer
Abspaltung des LV Bayerns von der Piratenpartei Deutschland. Dieser Antrag hingegen will
lediglich ein Landesprogramm ermöglichen, welches das Bundesprogramm ergänzt und nicht
ersetzt.
