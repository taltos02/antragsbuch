\anderungsantrag
% Antragsnummer
{S03}
% Antragstitel
{2/3-Mehrheit für Positionspapiere}
% Wikiurl zum Antrag
{http://wiki.piratenpartei.de/BY:Landesparteitag 2012.1/Antragsfabrik/Zweidrittelmehrheit_für_Positionspapiere}
% Antragssteller
{Fard}
% Betrifft
{Satzung des Landesverbands Bayern / § 11}
\subsection{Beantragte Änderung}
Es wird beantragt, in der Satzung Abschnitt §11 einen neuen Absatz 3 einzufügen. Der bisherige
Absatz 3 wird dann Absatz 4.\par
(3) Die Regelungen aus Absatz 1 und 2 gelten ebenso für die Annahme von Positionspapieren des
Landesverbandes Bayern der Piratenpartei Deutschland.
\subsection{Begründung}
Die Satzungsänderung bewirkt, dass auch Positionspapiere eine 2/3-Mehrheit benötigen.
Aus folgenden 2 Gründen schlage ich dies vor:
\begin{itemize}
	\item Das aktuelle Konzepts für die Erstellung des Wahlprogramms (siehe BY:Programmentwicklung Bayern) sieht folgendes vor: Positionspapiere werden mit 50\% beschlossen und aus den Inhalten dieser Positionspapiere wird ein Wahlprogramm erstellt. Dieses benötigt jedoch eine 2/3-Mehrheit. Ich sehe hier massiv Probleme auf uns zukommen. Positionspapiere, deren Inhalt unser Wahlprogramm bilden sollen, müssen auch mit 2/3-Mehrheit beschlossen werden.
	\item Es zeigt sich, dass Entscheidungen mit knapper Mehrheit sehr viel Unruhe innerhalb der Partei erzeugen können. Daher schlage ich eine 2/3-Mehrheit auch für Positionspapiere vor.
\end{itemize}

Auch wenn Positionspapiere und Programmanträge eine 2/3 Mehrheit benötigen, gibt es noch
folgende Unterschiede:
\begin{itemize}
	\item Inhalte, welche zu lang oder zu detailreich sind um in das Programm aufgenommen zu werden, können als Positionspapiere beschlossen werden.
	\item Programmanträge sollen mehr oder weniger in der beschlossenen Form in das Programm eingebaut werden. Bei Positionspapieren kann nur der Inhalt beschlossen werden, die Formulierung kann danach noch deutlich geändert werden (Z.B. um aus Positionspapieren ein gutes Wahlprogramm zu entwerfen).
\end{itemize}
Die Satzungsänderung ist von dem Antrag von Magnus für den BPT 2011.2 geklaut.
