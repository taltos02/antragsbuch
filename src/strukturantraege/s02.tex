\anderungsantrag
% Antragsnummer
{S02}
% Antragstitel
{Grundsatzprogramm des LV Bayern}
% Wikiurl zum Antrag
{http://wiki.piratenpartei.de/BY:Landesparteitag 2012.1/Antragsfabrik/Grundsatzprogramm_des_LV_Bayern}
% Antragssteller
{Boris Turovskiy}
% Betrifft
{Satzung des Landesverbands Bayern / §11(3)}
\subsection{Beantragte Änderung}
§11(3) der Satzung des Landesverbands Bayern wird wie folgt neugefasst:\\
\textbf{1Vom Landesparteitag kann ein eigenes Grundsatzprogramm für den Landesverband sowie
Wahlprogramme für Landtagswahlen verabschiedet werden. 2Diese dürfen dem
Grundsatzprogramm der Piratenpartei Deutschland nicht widersprechen.}
\subsection{Begründung}
Zuerst eine Übersicht der Änderungen:\\
\emph{Bisherige Version:}
\textit{
1Der Landesverband übernimmt das Grundsatzprogramm der Piratenpartei Deutschland. 2Vom
Landesparteitag kann ein eigenes Wahlprogramm für Kommunal- und Landtagswahlen
verabschiedet werden. 3Dieses muss auf den Werten des Grundsatzprogrammes basieren.}
Es wird also
\begin{itemize}
	\item Der 1. Satz (Übernahme des Bundesgrundsatzprogramms) gestrichen, dafür ein eigenes Landesgrundsatzprogramm eingeführt - der wichtigste Punkt, Erläutrungen dazu weiter unten;
	\item Im (vormals) 2. Satz zudem der Verweis auf Programme für Kommunalwahlen gestrichen (ist nicht Aufgabe des LVs/LPTs);
	\item Der letzte Satz wird abgeschwächt, statt des schwammigen und zugleich restriktiven \glqq Muss auf Werten basieren\grqq\  wird lediglich der direkte Widerspruch zum Bundesprogramm ausgeschlossen.
\end{itemize}
Nun zum eigentlichen Sinn des Antrags:\\

Die jetzige Fassung stammt aus einer Zeit, als weder die Größe noch die (auch regionale)
Inhomogenität der Piratenpartei absehbar waren. Das Konzept, dass die Landesverbände
programmatisch stets dem Bundesgrundsatzprogramm folgen müssen, wurde seitdem von allen
wahlkämpfenden LVs über Bord geworfen; Berlin hat auch ein eigenes Grundsatzprogramm
verabschiedet. Der LV Bayern - obwohl es der größte Landesverband ist - verhält sich bisher
äußerst zurückhaltend und brav, was uns und unseren Anliegen nicht gerade weiterhilft.
Daneben sehe ich mindestens drei weitere Vorteile, die aus der Satzungsänderung entstehen:
