\sonstigerantrag{X04}{Erstellung des Wahlprogramms}{http://wiki.piratenpartei.de/BY:Landesparteitag 2012.1/Antragsfabrik/Wahlprogramm}{CEdge}
\subsection{Antrag}
Der Landesparteitag möge folgendes beschließen:\\
Zur Entwicklung des Wahlprogramms können für den Landesparteitag Positionspapiere und
Programmanträge für das Wahlprogramm eingereicht werden.\par
Der Landesvorstand wird beauftragt, dafür zu sorgen, dass die Inhalte und Forderungen der
Positionspapiere in das Landeswahlprogramm einfließen. Die Texte der Wahlprogrammanträge
werden direkt in das Landeswahlprogramm übernommen.\par
Weiterhin wird der Landesvorstand beauftragt, für eine hinreichende redaktionelle und formelle
(Reihenfolge der Punkte) Ausarbeitung des Landeswahlprogramms zu sorgen. Dies soll zeitig nach
dem Abschluss der inhaltlichen Arbeit stattfinden, sodass das Wahlprogramm spätestens im
Frühjahr 2013 beschlossen werden kann.\par
Falls ein Landesgrundsatzprogramm beschlossen wird, können analog Programmanträge für dieses
gestellt werden. Für diesen Fall wird außerdem der Landesvorstand beauftragt, dafür zu sorgen,
dass dieses redaktionell angemessen aufbereitet wird.
\subsection{Begründung}
Nachdem das im September 2010 vom Landesparteitag beschlossene Konzept zur
Programmentwicklung nur in Teilen umgesetzt wurde, braucht es eine klärende Aussage des
Landesparteitags und einen Beschluss zum weiteren Vorgehen.\par
Einige Piraten haben bereits aufwändig Positionspapiere umgesetzt, in der Erwartung, dass deren
Inhalte wie zuvor beschlossen in das Wahlprogramm einfließen. Es wäre fatal, diese Arbeit zunichte
zu machen. Gleichzeitig haben wir Anträge zum Wahlprogramm vorliegen, die wir nicht einfach
außen vor lassen können und sollten. Also halten wir uns beide Wege offen.\par
So sollen auf den Parteitagen im März und fortfolgend Inhalte beschlossen werden, aus diesen
entsteht das Wahlprogramm. So haben wir ausreichend Zeit und die wesentlichen Inhalte des
Wahlprogramms stehen fest, bevor Listen und Direktkandidaten gewählt werden.\par
Im Antragstext steht bewusst die Formulierung, dass der Landesvorstand dafür Sorge tragen soll,
dass die Inhalte zu einem Landeswahlprogramm aufbereitet werden. Das muss nicht heißen, dass
die Vorstandsmitglieder selbst dies vornehmen müssen, sie tragen aber die Verantwortung dafür,
dass es geschieht.\par
Die Aussage zu einem eventuellen Landesgrundsatzprogramm ist, dass hierfür eigene
Programmanträge gestellt werden sollen und dieses Programm - falls es beschlossen wird -
ebenfalls aufbereitet werden soll, allerdings lediglich redaktionell (Rechtschreibfehler usw.).
