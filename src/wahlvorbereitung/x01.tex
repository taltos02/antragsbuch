\sonstigerantrag
% Antragsnummer
{X01}
% Antragstitel
{Wahlteilnahme}
% Wikiurl zum Antrag
{http://wiki.piratenpartei.de/BY:Landesparteitag 2012.1/Antragsfabrik/Aufstellung zur Wahl}
% Antragssteller
{Oliver T. Vaillant}

\subsection{Antrag}
Der ordentliche Parteitag des Landesverbands Bayern der Piratenpartei Deutschland möge
beschließen:
\begin{itemize}
	\item Bundeswahlen: Der Landesverband Bayern nimmt an den nächsten Wahlen zum Deutschen Bundestag mit eigenen Wahlvorschlägen teil. Der Vorstand des Landesverbands ist hiermit beauftragt und angewiesen, alle rechtlichen, organisatorischen und tatsächlichen Handlungen vorzunehmen, um die Wahlteilnahme des Landesverbands zu bewirken.
	\item Bayernwahlen: Der Landesverband Bayern nimmt an den nächsten Wahlen sowohl zum bay. Landtag als auch zu den bay. Bezirkstagen mit eigenen Wahlvorschlägen teil. Die Vorstandschaften der Bezirksverbände des Landesverbands sind hiermit beauftragt und angewiesen, alle rechtlichen, organisatorischen und tatsächlichen Handlungen vorzunehmen, um die Wahlteilnahme des Landesverbands und seiner Gliederungen zu bewirken, soweit sie nicht nach Recht und Gesetz dem Vorstand des Landesverbands vorbehalten sind.
\end{itemize}

\subsection{Begründung}
Das Wahlrecht ist geprägt von großer Formenstrenge; die Zulassung zu einer öffentlichen Wahl
erfordert daher eine Vielzahl streng formal geregelter Rechtsgeschäfte wie z.B.
\begin{itemize}
	\item Beteiligungsanzeigen (je eine für Bundestags- und Landtagswahlen, sieben für die Bezirkstagswahlen);
	\item Ladungen zu Kandidatenaufstellungen (46 für Bundeswahlen und mindestens 97 Versammlungen für die Bayernwahlen),
	\item  Dokumentation dieser Aufstellungsversammlungen,
	\item Einreichen der Wahlvorschläge (für Bayernwahlen sieben Wahlkreisvorschläge samt Unterstützungsunterschriften, für die Bundestagswahlen eine bay. Landesliste und 45 Direktkandidaten alias „Kreiswahlvorschläge“),
	\item und noch vieles anderes mehr.
\end{itemize}
In einigen Fällen sind diese reinen Formalia gesetzlich geregelt, doch für die meisten Fragen verweisen
die Wahlgesetze pauschal auf „die Satzung der Parteien“\footnote{1. das BWahlG z.B. in § 21 Abs.1 Satz 3 und Abs.5 BWahlG, das bay. LWG z.B. in Art.28 Abs.4 Abs.1 bay. LWG}; in unserer Satzung findet sich da aber nur
der § 10 in Abschnitt A der Satzung, der nur herzlich wenig aussagt, und deshalb haben wir ein
Zuständigkeitsproblem: Alle formalen Rechtsgeschäfte sind nur dann rechtsgültig, wenn sie von
jemand vorgenommen wurden, der dazu nach Recht und Gesetz auch ausdrücklich vertretungsbefugt
war, kurz: der das Dokument auch unterschreiben durfte.


Nimmt der LPT diesen Antrag jedoch an, dann sind die Vorstände schon durch den einfachen LPT-
Beschluss rechtlich ohne Weiteres befugt, die erforderlichen Rechtsgeschäfte vorzunehmen; der Antrag
besagt jedoch nichts über die tatsächliche Organisation der Kandidaten; dabei müssen wir uns dann nur
an das geltende Recht halten. Wird mein Antrag angenommen, dann fallen viele rein formale Gründe
weg, mit denen uns die Zulassungsausschüsse die Wahlzulassung verweigern könnten – und wir
schaffen es tatsächlich auf dem Wahlzettel.
