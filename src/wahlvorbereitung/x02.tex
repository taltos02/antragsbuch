\sonstigerantrag
% Antragsnummer
{X02}
% Antragstitel
{Wahlkampf}
% Wikiurl zum Antrag
{http://wiki.piratenpartei.de/BY:Landesparteitag 2012.1/Antragsfabrik/Wahlkampf}
% Antragssteller
{ander}

\subsection{Antrag}
Die Piratenpartei ist in den letzten Jahren wesentlich über das Internet aktiv geworden. Diese
Tatsache hat zu einer gegenüber den Altparteien deutlich erhöhten Kompetenz in Sachen
Internetkommunikation geführt. Jetzt, wo wir auf dem besten Wege zu einer Bürgerpartei sind, geht
es darum, auch internetferne Bevölkerungskreise insbesondere in Wahlkämpfen anzusprechen, zu
mobilisieren und für Wahl unserer Partei und Kandidaten zu gewinnen. Deshalb soll die
Wahlkampfarbeit über die bisherigen Gremien der Partei und das Internet ergänzt werden durch
\glqq Face to Face\grqq\  Veranstaltungen.\par
Bei allen Wahlkämpfen werden die Piraten in allen Groß- und Mittelstädten mit öffentlichen
Veranstaltungen in Erscheinung treten. Für Kleinstädte und ländliche Regionen sind adäquate
Formen öffentlichen Auftritts zu kreieren und zu praktizieren.\par
Der Parteitag beauftragt den Parteivorstand ein zentrales Wahlkampfgremium/Büro einzurichten,
das in Zusammenarbeit mit einer professionellen Werbeagentur die Wahlkämpfe der Piraten auf
allen Ebenen unterstützt. Die Mitglieder dieses Gremiums sollten im Rahmen freier
Mitarbeiterverträge angemessen bezahlt werden. Das Gremium bestimmt einen Vorsitzenden, der
Mitglied des Bundesvorstandes ist und/oder an diesen direkt berichtet.
\subsection{Begründung}
Es erscheint dringend notwendig, dass unsere Wahlkämpfe in hohem Maße professionalisiert
werden, damit wir auch internetferne Bevölkerungskreise als WählerInnen gewinnen. Öffentliche
Veranstaltungen haben einen hohen Aufmerksamkeitswert bei Bevölkerung und Medien weit über
den Veranstaltungsort hinaus. Dabei sind Veranstaltungen in Groß- und Míttelstädten nicht hoch
genug einzuschätzen. Insbesondere dann, wenn zusätzlich zu den politischen Inhalten
kulturpolitische Angebote gemacht werden können. Die Identifikation zwischen Partei, ihren
Vertretern und den Wählern dürfte durch persönliche Begegnungen wesentlich positiv beeinflusst
werden.\par
Natürlich ist alle Wahlkampfarbeit auf die aktive ehrenamtliche Mitarbeit aller Mitglieder und
Funktionsträger angewiesen. Diese dürfte aber bei den anstehenden Wahlkämpfen nicht ausreichen.
Deshalb ist auf eine professionelle Werbeagentur zurückzugreifen, die uns inhaltlich, Psychologisch
und werblich unterstützt.\par
Die anstehenden Wahlkämpfe erfordern auch von aktiven Piraten einen Fulltime-Job, der
ehrenamtlich nicht zu bewerkstelligen ist. Deshalb sollen die Mitglieder des zentralen
Wahlkampfteams angemessen bezahlt werden.\par
Da die Piratenpartei an der gesetzlich geregelten Wahlkampfkostenerstattung partizipieren wird,
dürfte die Finanzierung der genannten Aktivitäten gesichert sein.
