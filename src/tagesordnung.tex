\chapter{Tagesordnung}

\section{Samstag 25.03.12}
\begin{itemize}
	\item 10:30 Eröffnung der Versammlung durch den Vorsitzenden, Grußworte
	\item Abstimmung über Zulassung von Gästen sowie Übertragungen und Aufnahmen
	\item Wahl der Versammlungsleiter, Wahlleiter, Protokollanten und Rechnungsprüfer
	\item Beschluss der Tages- und Geschäftsordnung
	\item Behandlung des Antragsblocks "Strukturanträge"
	\item Behandlung des Antragsblocks "Wahlvorbereitung" \pageref{Wahlvorbereitung}
	\item Behandlung der Anträge gemäß der Reihenfolge in der topantrag23-Liste
	\item 19:00 Vertagung des Parteitags auf den 26.03.
\end{itemize}
\section{Sonntag 26.03.12}
\begin{itemize}
	\item 10:30 Wiedereröffnung der Versammlung durch den Versammlungsleiter
	\item Behandlung des Antragsblocks "Finanzanträge"
	\item Weiterbehandlung der topantrag23
	\item Behandlung des Antragsblocks "Metaanträge"
	\item Behandlung des Antragsblocks "Tier2"
	\item 18:00 Abschlussworte und Schließung der Versammlung durch den Vorsitzenden
\end{itemize}
\vfill
\hinweis{Über die endgültige Tagesordnung entscheidet allein der Parteitag.\\
Die letzte Entscheidung und Verantwortung über die Tagesordnung obliegt dem Parteitag.
}
\newpage

% ---------- Vorläufige Antragsblöcke

\section{Tagesordnung in vorläufiger Fassung: Antragsblöcke}
\begin{multicols}{2}
\subsubsection{Block \glqq Strukturanträge\grqq:}
\begin{itemize}
	\topantrag{S01}{Landesprogramm LV Bayern}
	\topantrag{S02}{Grundsatzprogramm LV Bayern}
	\topantrag{S03}{Zweidrittelmehrheit für Positionspapiere}
	\topantrag{X04}{Erstellung eines Wahlprogramms}
	\topantrag{X05}{Rücknahme Programmentwicklungskonzept}
	\topantrag{X06}{Altanträge als Positionspapiere}
	\topantrag{X07}{Redaktionskommission}
\end{itemize}
\subsection{Block \glqq Wahlvorbereitung\grqq:}
\begin{itemize}
	\topantrag{X01}{Wahlteilnahme}
	\topantrag{X02}{Wahlkampfzentrale}
	\topantrag{X03}{Termin des LPT zur Aufstellung der Liste für die Bundestagswahl}
\end{itemize}
\subsection{topantrag23-Anträge:}
\begin{itemize}
	\topantrag{A13}{Schutz der Privatsphäre}
	\topantrag{A07}{Hochschulpolitik und Forschung}
	\topantrag{A14}{Volksentscheide}
	\topantrag{A09}{Renten- und Krankenversicherungssystem}
	\topantrag{A22}{Strominfrastruktur}
	\topantrag{A11}{Öffentlich-rechtlicher Rundfunk}
	\topantrag{A05}{ÖPNV/Mobilität}
	\topantrag{A06}{Grundeinkommen/ReSET/NESt}
	\topantrag{A18}{Polizei}
	\topantrag{A04}{Wirtschaftspaket (Kammernpflicht, Ladenschlussgesetz)}
	\topantrag{A17}{Freie Software und Dateiformate in staatlichen Institutionen}
	\topantrag{A08}{Finanzpaket}
	\topantrag{A20}{Drogenpolitik}
	\topantrag{A12}{Unabhängige Staatsanwaltschaften}
	\topantrag{A01}{Videoüberwachung}
	\topantrag{A23}{Verfassungsschutz}
	\topantrag{A16}{Wahlsystem}
	\topantrag{A19}{Tierschutz}
	\topantrag{A10}{Altenpflege}
	\topantrag{A15}{Waffenrecht}
	\topantrag{A02}{Doppik vs. Kameralistik}
	\topantrag{P48}{GEMA-Reform}
	\topantrag{P43}{Abschaffung der Kindergartengebühren}
\end{itemize}
\subsection{Block \glqq Finanzanträge\grqq\ :}
\begin{itemize}
	\topantrag{S08}{Verteilungsschlüssel staatlicher Mittel}
	\topantrag{S09}{Neuregelung der Mittelverwendung}
	\topantrag{S10}{Verteilung der Parteienfinanzierung auf Landesverband und Bezirksverbände}
	\topantrag{S11}{Verteilungsschlüssel Parteienfinanzierung in die Satzung}
	\topantrag{S07}{Verteilungsschlüssel der Mitgliedsbeiträge}
\end{itemize}
\subsection{Block \glqq Metaanträge\grqq\ :}
\begin{itemize}
	\topantrag{S04}{Antragstagung}
	\topantrag{X09}{Zusatz zur Antragstagung}
	\topantrag{S05}{Mitgliederentscheid}
	\topantrag{S06}{Satzungsänderung per Mitgliederentscheid}
	\topantrag{X08}{Priorisierung von Anträgen durch demokratische Vorlegitimation}
	\topantrag{X10}{Freie Gewissensentscheidung von Piratenabgeordneten}
\end{itemize}
\subsection{Block \glqq Tier2\grqq\ :}
\begin{itemize}
	\topantrag{P29}{Breitbandausbau in Bayern, digitale Dividende}
	\topantrag{PA38}{Für eine zukunftssichere Energiewirtschaft}
	\topantrag{PA45}{Bildung / Verfügbarkeit von Lerninhalten}
	\topantrag{P21}{Bauernhöfe statt Agrarfabriken}
	\topantrag{P25}{Begrenzung der Klassengrößen im Primär- und Sekundärbereich}
	\topantrag{P19}{Korruption}
	\topantrag{P11}{Konsequentes Einhalten von Menschenrechts- und Völkerrechtscharta}
	\topantrag{PA46}{Informationsfreiheitsgesetz}
	\topantrag{P01}{Positionspapier zum Verhältnis von Staat und Religion}
	\topantrag{P28}{Livestreams von Stadtraatsitzungen}
	\topantrag{P47}{Neues Schulsystem}
	\topantrag{P02}{Grundsätze unserer Bildungspolitik}
	\topantrag{P35}{Informationsrecht bei geheimen Maßnahmen}
	\topantrag{P24}{Landesbeauftragter für Bildung}
	\item danach weiter nach Liquidizer- Reihenfolge
\end{itemize}
\subsection{Wildcard}
\textit{Hinweis: Da es sich um einen Programmparteitag
handelt, werden Satzungsänderungs- und
sonstige Anträge, sofern sie nicht von
entscheidender Bedeutung für die
Programmatik oder die Arbeit des
Landesverbandes sind, nicht behandelt. Diese
Anträge werden aber auf eine "Wildcard"
gesetzt, die immer zum Zug kommt, wenn
unvorhergesehene Verzögerungen im
Parteitagsablauf, Unklarheiten über die
weitere Reihenfolge, Auszählpausen o.Ä.
auftreten, kurzum: wenn ungenutzte Zeit zur
Verfügung steht. Die nachfolgend aufgeführte
Reihenfolge ist willkürlich durch Mitglieder
der Antragskommission festgelegt, aber
empfehlenswert.
}
\begin{itemize}
	\topantrag{X16}{Holodeck Kodex (geblockt mit "Holodeck Leserecht")}
	\topantrag{X17}{Holodeck Leserecht (geblockt mit "Holodeck Kodex")}
	\topantrag{X14}{Begrenzung der Mandatszeit für Mitglieder der Piratenpartei}
	\topantrag{X15}{Nachhaltiger Sprachgebrauch}
	\topantrag{X11}{Transparenz über alles}
	\topantrag{X12}{Störtebecker Stiftung}
	\topantrag{X13}{Verzicht auf Parteiausschlussverfahren (PAV)}
	\topantrag{S12}{Schutz vor Unterwanderung durch unerwünschte Organisationen (Satzungsänderungsantrag)}
	\topantrag{S13}{Einführung eines Crew-Prinzips nach baierischer Prägung (Satzungsänderungsantrag)}
\end{itemize}
\end{multicols}
