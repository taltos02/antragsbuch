\section{Hinweise zur Geschäftsordnung (GO)}

\begin{quote}
Jeder Pirat kann jederzeit durch Heben beider Hände das Vorhaben anzeigen, einen Antrag zur Geschäftsordnung stellen zu wollen. Solch einer Wortmeldung ist nach der aktuellen Wortmeldung Vorrang zu geben. (GO \S 6.5. Abs. 1)
\end{quote}

\hinweis{Die Geschäftsordnung des bayerischen Landesparteitags und die Satzung des Landesverbandes Bayern finden sich im Wiki:
\begin{itemize}
	\item \url{http://wiki.piratenpartei.de/BY:Geschäftsordnung_des_bayrischen_Landesparteitags}
	\item \url{http://wiki.piratenpartei.de/BY:Satzung_des_Landesverband_Bayern}
\end{itemize}
}

Nach der GO kann jeder akkreditierte Pirat folgende GO-Anträge an den Landesparteitag stellen (kein Anspruch auf Vollständigkeit):
\begin{description}
	\item[Geheime Abstimmung] Der Antrag fordert die geheime Abstimmung oder Wahl. (GO \S 4. Abs. 3)
	\item[Auszählung] Der Antrag fordert eine genaue Auszählung bei unklaren Verhältnissen in einer Wahl oder Abstimmung. (GO \S 4. Abs. 5)
	\item[Wiederholung der Wahl/Abstimmung] Die Versammlung kann die Wiederholung einer Wahl oder Abstimmung fordern. (GO \S 4. Abs. 8)
	\item[Alternativantrag] Jeder Pirat kann einen Alternativantrag zu einem gestellten Antrag stellen. (GO \S 6.5. Abs. 2)
	\item[Ende der Rednerliste] Schließt die Rednerliste nach dem letzten derzeit am Mikrofon stehenden Redner. (GO \S 6.5.1. Abs. 1)
	\item[Änderung der Tagesordnung] Änderung der Tagesordnung. Dieser Antrag muss schriftlich gestellt werden. (GO \S 6.5.2. Abs. 1)
	\item[Änderung der Geschäftsordnung] Die Änderung der Geschäftsordnung muss die Änderung im Wortlaut aufführen. Dieser Antrag muss schriftlich gestellt werden. (GO \S 6.5.3. Abs. 1)
	\item[Einholung eines Meinungsbildes] Dieser Antrag bittet die Versammlung um ein Meinungsbild zu einer Frage. Das Meinungsbild kann nicht ausgezählt werden. (GO \S 6.5.4. Abs. 1)
	\item[Vertagung der Sitzung] Der Antrag muss den gewünschten Zeitpunkt (Tag und Uhrzeit) der Fortsetzung enthalten. (GO \S 6.5.5. Abs. 1)
	\item[Unterbrechung der Sitzung] Der Antrag muss die gewünschte Dauer (in Minuten) enthalten. (GO \S 6.5.6. Abs. 1)
	\item[Begrenzung der Redezeit] Mit diesem Antrag kann man die Redezeit aller Redner einschränken. Der Antrag muss die gewünschte maximale Dauer (in Sekunden) zukünftiger Redebeiträge enthalten und die Angabe machen, wie lange diese Beschränkung gelten soll. (GO \S 6.5.7. Abs. 1)
\end{description}

\section{Spezielle Wahlen und Abstimmungen}
\begin{quote}
Alle Abstimmungen und Wahlen finden grundsätzlich mit einfacher Mehrheit und offen statt, sofern nicht die Satzung, diese GO oder ein Gesetz ein anderes bestimmt. (GO \S 4. Abs. 1)
\end{quote}

\hinweis{Dies gilt nicht für
\begin{description}
	\item[Vorstandswahl] Diese Wahl ist geheim (kein Filmen, Streamen, Fotografieren!) (GO \S 5. Abs. 1). Gewählt ist der Kandidat, welcher die meisten Stimmen und eine absolute Mehrheit der sich nicht enthaltenden Abstimmenden erhält. (GO \S 5. Abs. 2)
	\item[Satzungsänderungsantrag] Dieser Antrag benötigt eine 2/3-Mehrheit zur Annahme der Änderungen. (Satzung \S 11. Abs. 1)
	\item[Programmantrag] Dieser Antrag benötigt eine 2/3-Mehrheit zur Annahme der Änderungen. (Satzung \S 11. Abs. 1)
\end{description}
}
