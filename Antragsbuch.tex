\documentclass[12pt,a4paper,oneside]{scrbook}
\usepackage[ngermanb]{babel}
\usepackage{amsmath}
\usepackage[mathletters]{ucs}
\usepackage[utf8]{inputenc}
\usepackage[T1]{fontenc}
\usepackage{graphicx}
\usepackage{wrapfig}
\setlength{\parindent}{0pt}
\setlength{\parskip}{6pt plus 2pt minus 1pt}
\usepackage{eurosym}
\usepackage{url}
\usepackage{fancyhdr}
\usepackage[left=2cm, right=2cm, top=2cm, bottom=2cm, includefoot, includehead]{geometry}
\usepackage{float}
\usepackage{multicol}
\usepackage{multirow}
\usepackage{tabularx}
\usepackage{lmodern}
 \usepackage{framed}
  \usepackage{color}
\usepackage[
	pdftex,
	bookmarksopen=true,
	bookmarksnumbered=true,
	colorlinks,
	linkcolor=black,
	citecolor=black,
	urlcolor=black
]{hyperref}
\usepackage{lastpage}

% ---------- Daten des Landesparteitages

\newcommand{\lpt}{
	Landesparteitag Bayern 2012.2
}

\newcommand{\datum}{
	15./16. September 2012
}

% ---------- Angaben zum Ort
\newcommand{\halle}{
	Stadthalle Maxhütte-Haidhof
}

\newcommand{\strasse}{
	Regensburger Straße 75
}

\newcommand{\plz}{
	93142
}

\newcommand{\ort}{
	Maxhütte Haidhof (Oberpfalz)
}

\usepackage{newcommands}

\hypersetup{pdfauthor = {Landesverband Bayern, Piratenpartei Deutschland}, pdftitle = {Antragsbuch zum \lpt}} 

\begin{document}

% ---------- Titelseite

\pagestyle{empty}
\hypertarget{Titelseite}{} % Marke für hyperref pdfbookmark
\pdfbookmark[1]{Titelseite}{Titelseite}
\begin{titlepage}
\begin{Large}
\textsc{
Piratenpartei Deutschland Landesverband Bayern
}
\end{Large}
\vspace*{4cm}
\begin{center}
\begin{Huge}
\textbf{
Antragsbuch
}
\end{Huge}
\\[.5cm]
\begin{huge}
\textbf{
\lpt
}
\end{huge}
\\[1cm]
\includegraphics{media/PIRATENsignet.png}
\\[1cm]
\begin{Large}
\textbf{\datum}
\end{Large}
\\[1cm]
\halle \\
\strasse \\
\plz \ort
\end{center}
\end{titlepage}
\clearpage

% ---------- Dedication


\pdfbookmark[1]{Inhaltsverzeichnis}{toc}
\tableofcontents
\newpage
\pagestyle{plain}

% ---------- Formales

\chapter{Hinweise und Formalien}
\section{Über das Antragsbuch}

In diesem Antragsbuch sind alle Anträge aus der Seite \glqq Antragsblöcke\grqq\footnote{
\url{http://wiki.piratenpartei.de/BY:Landesparteitag_2012.1/Antragsfabrik/Antragsbl\%C3\%B6cke}
} in der Version vom
22. März 2012, 13:38 Uhr
berücksichtigt.


Antragsarten:
\begin{description}
	\item[PA] Programmantrag
	\item[P] Positionspapier
	\item[S] Satzungsänderungsantrag
	\item[X]  Sonstiger Antrag
\end{description}


Bei Fragen zum Antragsbuch bitte eine E-Mail  an \url{haidefs@piratenpartei-bayern.de}, oder \url{vorstand@piratenpartei-bayern.de} schreiben.

\section{Akkreditierung und Stimmrecht}
Stimmberechtigt ist nur, wer akkreditiert wurde.

Die Akkreditierung startet um 10:00 Uhr.


\hinweis{Wichtig für Akkreditierung:
\begin{itemize}
	\item ein gültiges Ausweisdokument (Personalausweis oder Reisepass) 
	\item man darf mit seinen Beitragszahlungen nicht mehr als drei Monate im Rückstand sein 
\end{itemize}
}

Hilfreich ist außerdem:
\begin{itemize}
	\item ein Nachweis der Beitragszahlung, falls die Buchung noch nicht erfasst worden sein sollte
	\item ein Mitgliedsausweis (sofern schon erhalten)
\end{itemize}

\hinweis{Es ist möglich, seinen Mitgliedsbeitrag auf der Versammlung in bar zu entrichten.}


Über die Zulassung von Gästen, Presse, Livestream oder Aufzeichnung wird zum Beginn des Parteitages abgestimmt.\\
Haustiere sind am Parteitag nicht möglich.

\section{Hinweise zur Geschäftsordnung (GO)}

\begin{quote}
Jeder Pirat kann jederzeit durch Heben beider Hände das Vorhaben anzeigen, einen Antrag zur Geschäftsordnung stellen zu wollen. Solch einer Wortmeldung ist nach der aktuellen Wortmeldung Vorrang zu geben. (GO \S 6.5. Abs. 1)
\end{quote}

\hinweis{Die Geschäftsordnung des bayerischen Landesparteitags und die Satzung des Landesverbandes Bayern finden sich im Wiki:
\begin{itemize}
	\item \url{http://wiki.piratenpartei.de/BY:Geschäftsordnung_des_bayrischen_Landesparteitags}
	\item \url{http://wiki.piratenpartei.de/BY:Satzung_des_Landesverband_Bayern}
\end{itemize}
}

Nach der GO kann jeder akkreditierte Pirat folgende GO-Anträge an den Landesparteitag stellen (kein Anspruch auf Vollständigkeit):
\begin{description}
	\item[Geheime Abstimmung] Der Antrag fordert die geheime Abstimmung oder Wahl. (GO \S 4. Abs. 3)
	\item[Auszählung] Der Antrag fordert eine genaue Auszählung bei unklaren Verhältnissen in einer Wahl oder Abstimmung. (GO \S 4. Abs. 5)
	\item[Wiederholung der Wahl/Abstimmung] Die Versammlung kann die Wiederholung einer Wahl oder Abstimmung fordern. (GO \S 4. Abs. 8)
	\item[Alternativantrag] Jeder Pirat kann einen Alternativantrag zu einem gestellten Antrag stellen. (GO \S 6.5. Abs. 2)
	\item[Ende der Rednerliste] Schließt die Rednerliste nach dem letzten derzeit am Mikrofon stehenden Redner. (GO \S 6.5.1. Abs. 1)
	\item[Änderung der Tagesordnung] Änderung der Tagesordnung. Dieser Antrag muss schriftlich gestellt werden. (GO \S 6.5.2. Abs. 1)
	\item[Änderung der Geschäftsordnung] Die Änderung der Geschäftsordnung muss die Änderung im Wortlaut aufführen. Dieser Antrag muss schriftlich gestellt werden. (GO \S 6.5.3. Abs. 1)
	\item[Einholung eines Meinungsbildes] Dieser Antrag bittet die Versammlung um ein Meinungsbild zu einer Frage. Das Meinungsbild kann nicht ausgezählt werden. (GO \S 6.5.4. Abs. 1)
	\item[Vertagung der Sitzung] Der Antrag muss den gewünschten Zeitpunkt (Tag und Uhrzeit) der Fortsetzung enthalten. (GO \S 6.5.5. Abs. 1)
	\item[Unterbrechung der Sitzung] Der Antrag muss die gewünschte Dauer (in Minuten) enthalten. (GO \S 6.5.6. Abs. 1)
	\item[Begrenzung der Redezeit] Mit diesem Antrag kann man die Redezeit aller Redner einschränken. Der Antrag muss die gewünschte maximale Dauer (in Sekunden) zukünftiger Redebeiträge enthalten und die Angabe machen, wie lange diese Beschränkung gelten soll. (GO \S 6.5.7. Abs. 1)
\end{description}

\section{Spezielle Wahlen und Abstimmungen}
\begin{quote}
Alle Abstimmungen und Wahlen finden grundsätzlich mit einfacher Mehrheit und offen statt, sofern nicht die Satzung, diese GO oder ein Gesetz ein anderes bestimmt. (GO \S 4. Abs. 1)
\end{quote}

\hinweis{Dies gilt nicht für
\begin{description}
	\item[Vorstandswahl] Diese Wahl ist geheim (kein Filmen, Streamen, Fotografieren!) (GO \S 5. Abs. 1). Gewählt ist der Kandidat, welcher die meisten Stimmen und eine absolute Mehrheit der sich nicht enthaltenden Abstimmenden erhält. (GO \S 5. Abs. 2)
	\item[Satzungsänderungsantrag] Dieser Antrag benötigt eine 2/3-Mehrheit zur Annahme der Änderungen. (Satzung \S 11. Abs. 1)
	\item[Programmantrag] Dieser Antrag benötigt eine 2/3-Mehrheit zur Annahme der Änderungen. (Satzung \S 11. Abs. 1)
\end{description}
}

\section{Anfahrt}
\subsection{Bahn}
Vom Bahnhof in der Bahnhofstraße bis zur Regensburgerstraße in Maxhütte-Haidhof sind es etwa 1,5 km.

Aus dem Zug raus und die Bahnhofstraße rechts entlang für etwa 650m, 
dann links in die Friedrich-Ebert-Straße.\\
Am Ende der Straße kommt man direkt auf die Regensburger Straße, hier links abbiegen.\\
Die Stadthalle befindet sich auf der linken Seite nach etwa 250m.


% ---------- Tagesordnung und Vorläufige Antragsblöcke

\chapter{Tagesordnung}

\section{Samstag 25.03.12}
\begin{itemize}
	\item 10:30 Eröffnung der Versammlung durch den Vorsitzenden, Grußworte
	\item Abstimmung über Zulassung von Gästen sowie Übertragungen und Aufnahmen
	\item Wahl der Versammlungsleiter, Wahlleiter, Protokollanten und Rechnungsprüfer
	\item Beschluss der Tages- und Geschäftsordnung
	\item Behandlung des Antragsblocks "Strukturanträge"
	\item Behandlung des Antragsblocks "Wahlvorbereitung" \pageref{Wahlvorbereitung}
	\item Behandlung der Anträge gemäß der Reihenfolge in der topantrag23-Liste
	\item 19:00 Vertagung des Parteitags auf den 26.03.
\end{itemize}
\section{Sonntag 26.03.12}
\begin{itemize}
	\item 10:30 Wiedereröffnung der Versammlung durch den Versammlungsleiter
	\item Behandlung des Antragsblocks "Finanzanträge"
	\item Weiterbehandlung der topantrag23
	\item Behandlung des Antragsblocks "Metaanträge"
	\item Behandlung des Antragsblocks "Tier2"
	\item 18:00 Abschlussworte und Schließung der Versammlung durch den Vorsitzenden
\end{itemize}
\vfill
\hinweis{Über die endgültige Tagesordnung entscheidet allein der Parteitag.\\
Die letzte Entscheidung und Verantwortung über die Tagesordnung obliegt dem Parteitag.
}
\newpage

% ---------- Vorläufige Antragsblöcke

\section{Tagesordnung in vorläufiger Fassung: Antragsblöcke}
\begin{multicols}{2}
\subsubsection{Block \glqq Strukturanträge\grqq:}
\begin{itemize}
	\topantrag{S01}{Landesprogramm LV Bayern}
	\topantrag{S02}{Grundsatzprogramm LV Bayern}
	\topantrag{S03}{Zweidrittelmehrheit für Positionspapiere}
	\topantrag{X04}{Erstellung eines Wahlprogramms}
	\topantrag{X05}{Rücknahme Programmentwicklungskonzept}
	\topantrag{X06}{Altanträge als Positionspapiere}
	\topantrag{X07}{Redaktionskommission}
\end{itemize}
\subsection{Block \glqq Wahlvorbereitung\grqq:}
\begin{itemize}
	\topantrag{X01}{Wahlteilnahme}
	\topantrag{X02}{Wahlkampfzentrale}
	\topantrag{X03}{Termin des LPT zur Aufstellung der Liste für die Bundestagswahl}
\end{itemize}
\subsection{topantrag23-Anträge:}
\begin{itemize}
	\topantrag{A13}{Schutz der Privatsphäre}
	\topantrag{A07}{Hochschulpolitik und Forschung}
	\topantrag{A14}{Volksentscheide}
	\topantrag{A09}{Renten- und Krankenversicherungssystem}
	\topantrag{A22}{Strominfrastruktur}
	\topantrag{A11}{Öffentlich-rechtlicher Rundfunk}
	\topantrag{A05}{ÖPNV/Mobilität}
	\topantrag{A06}{Grundeinkommen/ReSET/NESt}
	\topantrag{A18}{Polizei}
	\topantrag{A04}{Wirtschaftspaket (Kammernpflicht, Ladenschlussgesetz)}
	\topantrag{A17}{Freie Software und Dateiformate in staatlichen Institutionen}
	\topantrag{A08}{Finanzpaket}
	\topantrag{A20}{Drogenpolitik}
	\topantrag{A12}{Unabhängige Staatsanwaltschaften}
	\topantrag{A01}{Videoüberwachung}
	\topantrag{A23}{Verfassungsschutz}
	\topantrag{A16}{Wahlsystem}
	\topantrag{A19}{Tierschutz}
	\topantrag{A10}{Altenpflege}
	\topantrag{A15}{Waffenrecht}
	\topantrag{A02}{Doppik vs. Kameralistik}
	\topantrag{P48}{GEMA-Reform}
	\topantrag{P43}{Abschaffung der Kindergartengebühren}
\end{itemize}
\subsection{Block \glqq Finanzanträge\grqq\ :}
\begin{itemize}
	\topantrag{S08}{Verteilungsschlüssel staatlicher Mittel}
	\topantrag{S09}{Neuregelung der Mittelverwendung}
	\topantrag{S10}{Verteilung der Parteienfinanzierung auf Landesverband und Bezirksverbände}
	\topantrag{S11}{Verteilungsschlüssel Parteienfinanzierung in die Satzung}
	\topantrag{S07}{Verteilungsschlüssel der Mitgliedsbeiträge}
\end{itemize}
\subsection{Block \glqq Metaanträge\grqq\ :}
\begin{itemize}
	\topantrag{S04}{Antragstagung}
	\topantrag{X09}{Zusatz zur Antragstagung}
	\topantrag{S05}{Mitgliederentscheid}
	\topantrag{S06}{Satzungsänderung per Mitgliederentscheid}
	\topantrag{X08}{Priorisierung von Anträgen durch demokratische Vorlegitimation}
	\topantrag{X10}{Freie Gewissensentscheidung von Piratenabgeordneten}
\end{itemize}
\subsection{Block \glqq Tier2\grqq\ :}
\begin{itemize}
	\topantrag{P29}{Breitbandausbau in Bayern, digitale Dividende}
	\topantrag{PA38}{Für eine zukunftssichere Energiewirtschaft}
	\topantrag{PA45}{Bildung / Verfügbarkeit von Lerninhalten}
	\topantrag{P21}{Bauernhöfe statt Agrarfabriken}
	\topantrag{P25}{Begrenzung der Klassengrößen im Primär- und Sekundärbereich}
	\topantrag{P19}{Korruption}
	\topantrag{P11}{Konsequentes Einhalten von Menschenrechts- und Völkerrechtscharta}
	\topantrag{PA46}{Informationsfreiheitsgesetz}
	\topantrag{P01}{Positionspapier zum Verhältnis von Staat und Religion}
	\topantrag{P28}{Livestreams von Stadtraatsitzungen}
	\topantrag{P47}{Neues Schulsystem}
	\topantrag{P02}{Grundsätze unserer Bildungspolitik}
	\topantrag{P35}{Informationsrecht bei geheimen Maßnahmen}
	\topantrag{P24}{Landesbeauftragter für Bildung}
	\item danach weiter nach Liquidizer- Reihenfolge
\end{itemize}
\subsection{Wildcard}
\textit{Hinweis: Da es sich um einen Programmparteitag
handelt, werden Satzungsänderungs- und
sonstige Anträge, sofern sie nicht von
entscheidender Bedeutung für die
Programmatik oder die Arbeit des
Landesverbandes sind, nicht behandelt. Diese
Anträge werden aber auf eine "Wildcard"
gesetzt, die immer zum Zug kommt, wenn
unvorhergesehene Verzögerungen im
Parteitagsablauf, Unklarheiten über die
weitere Reihenfolge, Auszählpausen o.Ä.
auftreten, kurzum: wenn ungenutzte Zeit zur
Verfügung steht. Die nachfolgend aufgeführte
Reihenfolge ist willkürlich durch Mitglieder
der Antragskommission festgelegt, aber
empfehlenswert.
}
\begin{itemize}
	\topantrag{X16}{Holodeck Kodex (geblockt mit "Holodeck Leserecht")}
	\topantrag{X17}{Holodeck Leserecht (geblockt mit "Holodeck Kodex")}
	\topantrag{X14}{Begrenzung der Mandatszeit für Mitglieder der Piratenpartei}
	\topantrag{X15}{Nachhaltiger Sprachgebrauch}
	\topantrag{X11}{Transparenz über alles}
	\topantrag{X12}{Störtebecker Stiftung}
	\topantrag{X13}{Verzicht auf Parteiausschlussverfahren (PAV)}
	\topantrag{S12}{Schutz vor Unterwanderung durch unerwünschte Organisationen (Satzungsänderungsantrag)}
	\topantrag{S13}{Einführung eines Crew-Prinzips nach baierischer Prägung (Satzungsänderungsantrag)}
\end{itemize}
\end{multicols}


% ---------- Anträge

\clearpage

\antragsblock{Wahlvorbereitung}

\sonstigerantrag
% Antragsnummer
{X01}
% Antragstitel
{Wahlteilnahme}
% Wikiurl zum Antrag
{http://wiki.piratenpartei.de/BY:Landesparteitag 2012.1/Antragsfabrik/Aufstellung zur Wahl}
% Antragssteller
{Oliver T. Vaillant}

\subsection{Antrag}
Der ordentliche Parteitag des Landesverbands Bayern der Piratenpartei Deutschland möge
beschließen:
\begin{itemize}
	\item Bundeswahlen: Der Landesverband Bayern nimmt an den nächsten Wahlen zum Deutschen Bundestag mit eigenen Wahlvorschlägen teil. Der Vorstand des Landesverbands ist hiermit beauftragt und angewiesen, alle rechtlichen, organisatorischen und tatsächlichen Handlungen vorzunehmen, um die Wahlteilnahme des Landesverbands zu bewirken.
	\item Bayernwahlen: Der Landesverband Bayern nimmt an den nächsten Wahlen sowohl zum bay. Landtag als auch zu den bay. Bezirkstagen mit eigenen Wahlvorschlägen teil. Die Vorstandschaften der Bezirksverbände des Landesverbands sind hiermit beauftragt und angewiesen, alle rechtlichen, organisatorischen und tatsächlichen Handlungen vorzunehmen, um die Wahlteilnahme des Landesverbands und seiner Gliederungen zu bewirken, soweit sie nicht nach Recht und Gesetz dem Vorstand des Landesverbands vorbehalten sind.
\end{itemize}

\subsection{Begründung}
Das Wahlrecht ist geprägt von großer Formenstrenge; die Zulassung zu einer öffentlichen Wahl
erfordert daher eine Vielzahl streng formal geregelter Rechtsgeschäfte wie z.B.
\begin{itemize}
	\item Beteiligungsanzeigen (je eine für Bundestags- und Landtagswahlen, sieben für die Bezirkstagswahlen);
	\item Ladungen zu Kandidatenaufstellungen (46 für Bundeswahlen und mindestens 97 Versammlungen für die Bayernwahlen),
	\item  Dokumentation dieser Aufstellungsversammlungen,
	\item Einreichen der Wahlvorschläge (für Bayernwahlen sieben Wahlkreisvorschläge samt Unterstützungsunterschriften, für die Bundestagswahlen eine bay. Landesliste und 45 Direktkandidaten alias „Kreiswahlvorschläge“),
	\item und noch vieles anderes mehr.
\end{itemize}
In einigen Fällen sind diese reinen Formalia gesetzlich geregelt, doch für die meisten Fragen verweisen
die Wahlgesetze pauschal auf „die Satzung der Parteien“\footnote{1. das BWahlG z.B. in § 21 Abs.1 Satz 3 und Abs.5 BWahlG, das bay. LWG z.B. in Art.28 Abs.4 Abs.1 bay. LWG}; in unserer Satzung findet sich da aber nur
der § 10 in Abschnitt A der Satzung, der nur herzlich wenig aussagt, und deshalb haben wir ein
Zuständigkeitsproblem: Alle formalen Rechtsgeschäfte sind nur dann rechtsgültig, wenn sie von
jemand vorgenommen wurden, der dazu nach Recht und Gesetz auch ausdrücklich vertretungsbefugt
war, kurz: der das Dokument auch unterschreiben durfte.


Nimmt der LPT diesen Antrag jedoch an, dann sind die Vorstände schon durch den einfachen LPT-
Beschluss rechtlich ohne Weiteres befugt, die erforderlichen Rechtsgeschäfte vorzunehmen; der Antrag
besagt jedoch nichts über die tatsächliche Organisation der Kandidaten; dabei müssen wir uns dann nur
an das geltende Recht halten. Wird mein Antrag angenommen, dann fallen viele rein formale Gründe
weg, mit denen uns die Zulassungsausschüsse die Wahlzulassung verweigern könnten – und wir
schaffen es tatsächlich auf dem Wahlzettel.

\sonstigerantrag
% Antragsnummer
{X02}
% Antragstitel
{Wahlkampf}
% Wikiurl zum Antrag
{http://wiki.piratenpartei.de/BY:Landesparteitag 2012.1/Antragsfabrik/Wahlkampf}
% Antragssteller
{ander}

\subsection{Antrag}
Die Piratenpartei ist in den letzten Jahren wesentlich über das Internet aktiv geworden. Diese
Tatsache hat zu einer gegenüber den Altparteien deutlich erhöhten Kompetenz in Sachen
Internetkommunikation geführt. Jetzt, wo wir auf dem besten Wege zu einer Bürgerpartei sind, geht
es darum, auch internetferne Bevölkerungskreise insbesondere in Wahlkämpfen anzusprechen, zu
mobilisieren und für Wahl unserer Partei und Kandidaten zu gewinnen. Deshalb soll die
Wahlkampfarbeit über die bisherigen Gremien der Partei und das Internet ergänzt werden durch
\glqq Face to Face\grqq\  Veranstaltungen.\par
Bei allen Wahlkämpfen werden die Piraten in allen Groß- und Mittelstädten mit öffentlichen
Veranstaltungen in Erscheinung treten. Für Kleinstädte und ländliche Regionen sind adäquate
Formen öffentlichen Auftritts zu kreieren und zu praktizieren.\par
Der Parteitag beauftragt den Parteivorstand ein zentrales Wahlkampfgremium/Büro einzurichten,
das in Zusammenarbeit mit einer professionellen Werbeagentur die Wahlkämpfe der Piraten auf
allen Ebenen unterstützt. Die Mitglieder dieses Gremiums sollten im Rahmen freier
Mitarbeiterverträge angemessen bezahlt werden. Das Gremium bestimmt einen Vorsitzenden, der
Mitglied des Bundesvorstandes ist und/oder an diesen direkt berichtet.
\subsection{Begründung}
Es erscheint dringend notwendig, dass unsere Wahlkämpfe in hohem Maße professionalisiert
werden, damit wir auch internetferne Bevölkerungskreise als WählerInnen gewinnen. Öffentliche
Veranstaltungen haben einen hohen Aufmerksamkeitswert bei Bevölkerung und Medien weit über
den Veranstaltungsort hinaus. Dabei sind Veranstaltungen in Groß- und Míttelstädten nicht hoch
genug einzuschätzen. Insbesondere dann, wenn zusätzlich zu den politischen Inhalten
kulturpolitische Angebote gemacht werden können. Die Identifikation zwischen Partei, ihren
Vertretern und den Wählern dürfte durch persönliche Begegnungen wesentlich positiv beeinflusst
werden.\par
Natürlich ist alle Wahlkampfarbeit auf die aktive ehrenamtliche Mitarbeit aller Mitglieder und
Funktionsträger angewiesen. Diese dürfte aber bei den anstehenden Wahlkämpfen nicht ausreichen.
Deshalb ist auf eine professionelle Werbeagentur zurückzugreifen, die uns inhaltlich, Psychologisch
und werblich unterstützt.\par
Die anstehenden Wahlkämpfe erfordern auch von aktiven Piraten einen Fulltime-Job, der
ehrenamtlich nicht zu bewerkstelligen ist. Deshalb sollen die Mitglieder des zentralen
Wahlkampfteams angemessen bezahlt werden.\par
Da die Piratenpartei an der gesetzlich geregelten Wahlkampfkostenerstattung partizipieren wird,
dürfte die Finanzierung der genannten Aktivitäten gesichert sein.

\sonstigerantrag
% Antragsnummer
{X03}
% Antragstitel
{Termin des LPT zur Aufstellung der Landesliste für die Bundestagswahl}
% Wikiurl zum Antrag
{http://wiki.piratenpartei.de/BY:Landesparteitag 2012.1/Antragsfabrik/LPT_Liste_BTW}
% Antragssteller
{Oliver T. Vaillant}

\subsection{Antrag}
Der Landesparteitag möge beschließen:\\
Der Landesparteitag zur Aufstellung der Landesliste für die Bundestagswahl soll nach den
Sommerferien stattfinden.

\subsection{Begründung}
Der frühste Termin zur Aufstellung von Kandidaten zur Bundestagswahl ist der 28.6.2012 Der
Landesvorstand hat beschlossen den Parteitag zur Aufstellung der Landesliste Juni/Juli 2012
durchzuführen. Im Extremfall ist dann kein einziger Direktkandidat gewählt worden, bevor die
Liste aufgestellt wird. Für viele Piraten ist die Wahl der Kandidaten vor Ort eine wichtiges
Qualitätskriterium, auf das sie bei der Aufstellung der Landesliste nicht verzichten wollen.
Deswegen sollen im Juni/Juli 2012 zunächst die Direktkandidaten in ihren Wahlkreisen gewählt
werden und nach den Sommerferien dann die Landesliste.\par
Das Argument, bei Formfehlern bräuchte man genug Zeit um Notfalls den LPT wiederholen zu
können, ist angesichts der Fristen nicht schlüssig.\par
Da die Ausschreibung eh bis Mitte April geht, kann der LPT das nun entscheiden.


\chapter{Strukturanträge}\label{Strukturantraege} % Umlaute im label Befehl gehen nicht!

\anderungsantrag
% Antragsnummer
{S01}
% Antragstitel
{Landesprogramm des LV Bayern}
% Wikiurl zum Antrag
{http://wiki.piratenpartei.de/BY:Landesparteitag 2012.1/Antragsfabrik/Landesprogramm des LV Bayern}
% Antragssteller
{Fard}
% Betrifft
{Satzung des Landesverbands Bayern / §11(3)}
\konkurrenz{S02}
\subsection{Beantragte Änderung}
§11(3) der Satzung des Landesverbands Bayern wird wie folgt neugefasst:\\
1 Der Landesverband übernimmt das Grundsatzprogramm der Piratenpartei Deutschland. 2 Vom
Landesparteitag kann ein eigenes Landesprogramm für den Landesverband Bayern sowie
Wahlprogramme für Landtagswahlen verabschiedet werden. 3 Diese müssen auf den Werten des
Grundsatzprogrammes basieren.
\subsection{Begründung}
\subsubsection{Zuerst eine Übersicht der Änderungen:}
\begin{quotation}
1Der Landesverband übernimmt das Grundsatzprogramm der Piratenpartei Deutschland. 2Vom
Landesparteitag kann ein eigenes Wahlprogramm für Kommunal- und Landtagswahlen
verabschiedet werden. 3Dieses muss auf den Werten des Grundsatzprogrammes basieren.
\end{quotation}
Im 2. Satz wird der Verweis auf Programme für Kommunalwahlen gestrichen (ist nicht Aufgabe des
LVs/LPTs). Ausserdem wird die Möglichkeit geschaffen ein eigenes Landesprogramm zu erstellen.
Dies war bisher nicht vorgesehen.
\subsubsection{Nun zum eigentlichen Sinn des Antrags:}
Die sehr schleppend vorangehende Programmarbeit im LV ist zum Teil darauf zurückzuführen, dass
die Landesebene garnicht als \glqq Programmebne\grqq\  wahrgenommen wird. Auf Landesebene war bisher
das beste mögliche Schicksal für einen Programmantrag dass er in den Untiefen des Wikis als
\glqq Positionspapier\grqq\  Staub sammelt. Speziell für landespolitische Themen (Bildung, Polizei, etc.)
macht es Sinn neben konkreten Inhalten i

\subsubsection{Unterschied zum Antrag}
\url{http://wiki.piratenpartei.de/BY:Landesparteitag_2012.1/Antragsfabrik/Grundsatzprogramm_des_LV_Bayern}:\\
Im Gegensatz zum Antrag \glqq Grundsatzprogramm des LV Bayern\grqq\  bleibt bei diesem Antrag der
erste Satz {\quote Der Landesverband übernimmt das Grundsatzprogramm der Piratenpartei
Deutschland.}  in der Satzung. Falls der Antrag von Boris angenommen wird, möchte dieser das
aktuelle Bundesprogramm einzeln auf dem LPT abstimmen lassen. Dabei würde ein
Landesgrundsatzprogramm entstehen, welches nur noch die im LV Bayern mehrheitsfähigen
Positionen enthält. Dieses Landesgrundsatzprogramm wäre somit eine alternative bayerische
Version des Bundesprogrammes. Dies ist meiner Meinung nach ein erster Schritt hin zu einer
Abspaltung des LV Bayerns von der Piratenpartei Deutschland. Dieser Antrag hingegen will
lediglich ein Landesprogramm ermöglichen, welches das Bundesprogramm ergänzt und nicht
ersetzt.

\anderungsantrag
% Antragsnummer
{S02}
% Antragstitel
{Grundsatzprogramm des LV Bayern}
% Wikiurl zum Antrag
{http://wiki.piratenpartei.de/BY:Landesparteitag 2012.1/Antragsfabrik/Grundsatzprogramm_des_LV_Bayern}
% Antragssteller
{Boris Turovskiy}
% Betrifft
{Satzung des Landesverbands Bayern / §11(3)}
\subsection{Beantragte Änderung}
§11(3) der Satzung des Landesverbands Bayern wird wie folgt neugefasst:\\
\textbf{1Vom Landesparteitag kann ein eigenes Grundsatzprogramm für den Landesverband sowie
Wahlprogramme für Landtagswahlen verabschiedet werden. 2Diese dürfen dem
Grundsatzprogramm der Piratenpartei Deutschland nicht widersprechen.}
\subsection{Begründung}
Zuerst eine Übersicht der Änderungen:\\
\emph{Bisherige Version:}
\textit{
1Der Landesverband übernimmt das Grundsatzprogramm der Piratenpartei Deutschland. 2Vom
Landesparteitag kann ein eigenes Wahlprogramm für Kommunal- und Landtagswahlen
verabschiedet werden. 3Dieses muss auf den Werten des Grundsatzprogrammes basieren.}
Es wird also
\begin{itemize}
	\item Der 1. Satz (Übernahme des Bundesgrundsatzprogramms) gestrichen, dafür ein eigenes Landesgrundsatzprogramm eingeführt - der wichtigste Punkt, Erläutrungen dazu weiter unten;
	\item Im (vormals) 2. Satz zudem der Verweis auf Programme für Kommunalwahlen gestrichen (ist nicht Aufgabe des LVs/LPTs);
	\item Der letzte Satz wird abgeschwächt, statt des schwammigen und zugleich restriktiven \glqq Muss auf Werten basieren\grqq\  wird lediglich der direkte Widerspruch zum Bundesprogramm ausgeschlossen.
\end{itemize}
Nun zum eigentlichen Sinn des Antrags:\\

Die jetzige Fassung stammt aus einer Zeit, als weder die Größe noch die (auch regionale)
Inhomogenität der Piratenpartei absehbar waren. Das Konzept, dass die Landesverbände
programmatisch stets dem Bundesgrundsatzprogramm folgen müssen, wurde seitdem von allen
wahlkämpfenden LVs über Bord geworfen; Berlin hat auch ein eigenes Grundsatzprogramm
verabschiedet. Der LV Bayern - obwohl es der größte Landesverband ist - verhält sich bisher
äußerst zurückhaltend und brav, was uns und unseren Anliegen nicht gerade weiterhilft.
Daneben sehe ich mindestens drei weitere Vorteile, die aus der Satzungsänderung entstehen:

\anderungsantrag
% Antragsnummer
{S03}
% Antragstitel
{2/3-Mehrheit für Positionspapiere}
% Wikiurl zum Antrag
{http://wiki.piratenpartei.de/BY:Landesparteitag 2012.1/Antragsfabrik/Zweidrittelmehrheit_für_Positionspapiere}
% Antragssteller
{Fard}
% Betrifft
{Satzung des Landesverbands Bayern / § 11}
\subsection{Beantragte Änderung}
Es wird beantragt, in der Satzung Abschnitt §11 einen neuen Absatz 3 einzufügen. Der bisherige
Absatz 3 wird dann Absatz 4.\par
(3) Die Regelungen aus Absatz 1 und 2 gelten ebenso für die Annahme von Positionspapieren des
Landesverbandes Bayern der Piratenpartei Deutschland.
\subsection{Begründung}
Die Satzungsänderung bewirkt, dass auch Positionspapiere eine 2/3-Mehrheit benötigen.
Aus folgenden 2 Gründen schlage ich dies vor:
\begin{itemize}
	\item Das aktuelle Konzepts für die Erstellung des Wahlprogramms (siehe BY:Programmentwicklung Bayern) sieht folgendes vor: Positionspapiere werden mit 50\% beschlossen und aus den Inhalten dieser Positionspapiere wird ein Wahlprogramm erstellt. Dieses benötigt jedoch eine 2/3-Mehrheit. Ich sehe hier massiv Probleme auf uns zukommen. Positionspapiere, deren Inhalt unser Wahlprogramm bilden sollen, müssen auch mit 2/3-Mehrheit beschlossen werden.
	\item Es zeigt sich, dass Entscheidungen mit knapper Mehrheit sehr viel Unruhe innerhalb der Partei erzeugen können. Daher schlage ich eine 2/3-Mehrheit auch für Positionspapiere vor.
\end{itemize}

Auch wenn Positionspapiere und Programmanträge eine 2/3 Mehrheit benötigen, gibt es noch
folgende Unterschiede:
\begin{itemize}
	\item Inhalte, welche zu lang oder zu detailreich sind um in das Programm aufgenommen zu werden, können als Positionspapiere beschlossen werden.
	\item Programmanträge sollen mehr oder weniger in der beschlossenen Form in das Programm eingebaut werden. Bei Positionspapieren kann nur der Inhalt beschlossen werden, die Formulierung kann danach noch deutlich geändert werden (Z.B. um aus Positionspapieren ein gutes Wahlprogramm zu entwerfen).
\end{itemize}
Die Satzungsänderung ist von dem Antrag von Magnus für den BPT 2011.2 geklaut.

\sonstigerantrag{X04}{Erstellung des Wahlprogramms}{http://wiki.piratenpartei.de/BY:Landesparteitag 2012.1/Antragsfabrik/Wahlprogramm}{CEdge}
\subsection{Antrag}
Der Landesparteitag möge folgendes beschließen:\\
Zur Entwicklung des Wahlprogramms können für den Landesparteitag Positionspapiere und
Programmanträge für das Wahlprogramm eingereicht werden.\par
Der Landesvorstand wird beauftragt, dafür zu sorgen, dass die Inhalte und Forderungen der
Positionspapiere in das Landeswahlprogramm einfließen. Die Texte der Wahlprogrammanträge
werden direkt in das Landeswahlprogramm übernommen.\par
Weiterhin wird der Landesvorstand beauftragt, für eine hinreichende redaktionelle und formelle
(Reihenfolge der Punkte) Ausarbeitung des Landeswahlprogramms zu sorgen. Dies soll zeitig nach
dem Abschluss der inhaltlichen Arbeit stattfinden, sodass das Wahlprogramm spätestens im
Frühjahr 2013 beschlossen werden kann.\par
Falls ein Landesgrundsatzprogramm beschlossen wird, können analog Programmanträge für dieses
gestellt werden. Für diesen Fall wird außerdem der Landesvorstand beauftragt, dafür zu sorgen,
dass dieses redaktionell angemessen aufbereitet wird.
\subsection{Begründung}
Nachdem das im September 2010 vom Landesparteitag beschlossene Konzept zur
Programmentwicklung nur in Teilen umgesetzt wurde, braucht es eine klärende Aussage des
Landesparteitags und einen Beschluss zum weiteren Vorgehen.\par
Einige Piraten haben bereits aufwändig Positionspapiere umgesetzt, in der Erwartung, dass deren
Inhalte wie zuvor beschlossen in das Wahlprogramm einfließen. Es wäre fatal, diese Arbeit zunichte
zu machen. Gleichzeitig haben wir Anträge zum Wahlprogramm vorliegen, die wir nicht einfach
außen vor lassen können und sollten. Also halten wir uns beide Wege offen.\par
So sollen auf den Parteitagen im März und fortfolgend Inhalte beschlossen werden, aus diesen
entsteht das Wahlprogramm. So haben wir ausreichend Zeit und die wesentlichen Inhalte des
Wahlprogramms stehen fest, bevor Listen und Direktkandidaten gewählt werden.\par
Im Antragstext steht bewusst die Formulierung, dass der Landesvorstand dafür Sorge tragen soll,
dass die Inhalte zu einem Landeswahlprogramm aufbereitet werden. Das muss nicht heißen, dass
die Vorstandsmitglieder selbst dies vornehmen müssen, sie tragen aber die Verantwortung dafür,
dass es geschieht.\par
Die Aussage zu einem eventuellen Landesgrundsatzprogramm ist, dass hierfür eigene
Programmanträge gestellt werden sollen und dieses Programm - falls es beschlossen wird -
ebenfalls aufbereitet werden soll, allerdings lediglich redaktionell (Rechtschreibfehler usw.).

\sonstigerantrag{X05}{Rücknahme Programmentwicklungskonzept Bayern}
{http://wiki.piratenpartei.de/BY:Landesparteitag 2012.1/Antragsfabrik/Rücknahme_Programmentwicklungskonzept}
{Spencer}
\subsection{Antrag}
Der Landesparteitag beschließt, dass\\
das Programmentwicklungskonzept für Bayern (dass am 09.04.2010 als sonstiger Antrag
beschlossen wurde) zurückgenommen wird.
\subsection{Begründung}
Das Programmentwicklungskonzept wurde nicht umgesetzt und behindert - auch durch seinen
obsoleten zeitlichen Rahmen - die Programmatische Entwicklung. Es fanden weder Programmtage
statt noch wurden die Regelungen für IG, FG uns SG angenommen.\par
Das ursprüngliche, nicht umgesetzte Programmentwicklungskonzept
\url{http://wiki.piratenpartei.de/BY:Programmentwicklung_Bayern}

\sonstigerantrag{X06}{Altanträge als Positionspapiere}
{http://wiki.piratenpartei.de/BY:Landesparteitag 2012.1/Antragsfabrik/Altanträge_als_Positionspapiere}
{CEdge}
\subsection{Antrag}
Der Landesparteitag möge folgendes beschließen:\\
Die auf dem Landesparteitag 2010.1 beschlossenen Anträge \glqq Kennzeichnung von Polizeibeamten\grqq\ ,
\glqq Software in der öffentlichen Verwaltung\grqq\  und \glqq \glqq Neue\grqq\  Grundrechte\grqq\  werden, soweit nicht durch
andere Beschlüsse ersetzt, zu Positionspapieren umdefiniert.

\subsection{Begründung}
Diese Anträge entstanden vor der dem Programmentwicklungskonzept und waren für das
Landeswahlprogramm gedacht. Sie eignen sich ohne weiteres als Positionspapiere, da sie sehr
ausführlich sind und auch eine Begründung enthalten.\par
Durch den Beschluss als Positionspapier können die Inhalte der Beschlüsse ins Wahlprogramm
einfließen.\par
Die betroffenen Anträge stammen ausschließlich vom Urheber dieses Antrags. Dieser Antrag dürfte
eine 2/3-Mehrheit benötigen, da die umdeklarierten Programmanträge mit selbiger beschlossen
wurden.\par
Siehe auch\\
\url{http://wiki.piratenpartei.de/Landesverband_Bayern/Landeswahlprogramm}\\
\url{http://wiki.piratenpartei.de/Archiv:Antragsfabrik_Bayern/Identifikation_von_Polizeikräften}\\
\url{http://wiki.piratenpartei.de/Archiv:Antragsfabrik_Bayern/Software_in_der_öffentlichen_Verwaltung}\\
\url{http://wiki.piratenpartei.de/Archiv:Antragsfabrik_Bayern/"Neue"_Grundrechte}

\sonstigerantrag{X07}{Redaktionskommission}{http://wiki.piratenpartei.de/BY:Landesparteitag 2012.1/Antragsfabrik/Redaktionskommission}{TurBor}
\subsection{Antrag}
Der Landesparteitag möge sich dafür aussprechen, dass\\
bis zum nächsten Landesparteitag eine redaktionelle Bearbeitung des Parteiprogramms
durchgeführt wird. Dabei soll das Programm und die beschlossenen Positionspapiere - insbesondere
die auf dem Landesparteitag 2012.1 neu beschlossenen Punkte - klar und logisch strukturiert, von
sprachlichen Mängeln bereinigt und stilistisch einheitlich gestaltet werden. Es dürfen keine
inhaltlichen Veränderungen vorgenommen werden.\par
Die Überarbeitung wird durch eine durch den Vorstand oder den Landesparteitag einberufene
Programmkommission in Zusammenarbeit mit den Autoren der betroffenen Programmpunkte
durchgeführt.\par
Das so überarbeitete Programm bzw. Positionspapiere müssen, um Gültigkeit zu erlangen, durch
den nächsten Landesparteitag ratifiziert werden. Zu diesem Zweck möge der Landesvorstand die
vorgeschlagene Überarbeitung fristgerecht vor dem nächsten Landesparteitag zur parteiinternen
Diskussion stellen und einreichen.

\subsection{Begründung}
Sollte sogar ein geringer Anteil der vorgeschlagenen Programmänderungsanträge angenommen
werden, wird unser Parteiprogramm ziemlich chaotisch aussehen, und da die Anträge von sehr
vielen verschiedenen Personen stammen, ist weder eine einheitliche Struktur noch ein einheitlicher
Stil gewährleistet. Hinzu kommt, dass manche Anträge trotz inhaltlicher Stärke sprachlich schwach
sind.\par
Der Antrag zielt darauf ab, bereits auf dem LPT einen Übrarbeitungsvorgang einzuleiten, damit wir
auf dem nächsten Parteitag die überarbeitete Version ratifizieren können. Es handelt sich dabei nicht
um ein reines Korrekturlesen (Rechtschreib-/Grammatikfehler), da auch die Struktur sowie die
stilistischen Gegebenheiten geändert werden sollten. Der Inhalt muss natürlich in vollem Umfang
erhalten bleiben. Die eigentliche Arbeit wird wahrscheinlich von beauftragten Piraten durchgeführt,
der Vorstand ist aber für die Umsetzung verantwortlich.\par
Falls der Antrag angenommen wird, kann auch die Diskussion über alle nachfolgenden Anträge auf
dem Parteitag sich auf deren Inhalt und nicht eventuelle sprachliche Schwächen konzentrieren.



\positionspapier
% Antragsnummer
{P54}
% Antragstitel
{Doppik vs. Kameralistik A}
% Wikiurl zum Antrag
{http://wiki.piratenpartei.de/BY:Landesparteitag 2012.1/Antragsfabrik/Doppik+Kameralistik-A}
% Antragssteller
{Simon90L}
% Antragstext
\subsection{Antrag}
Der Landeshaushalt wird maßgeblich durch das
verwendete Rechnungswesen bestimmt. Bayern verwendet noch die Kameralistik
(Kameralbuchwesen), während einige Kommunen, Bundesländer und der Großteil der EU bereits
die Doppik (doppelten Buchführung) eingeführt haben. Die Kameralistik bietet keinen Überblick
über den Ressourcenverbrauch, eine Kosten- Leistungsrechnung kann nicht durchgängig
implementiert werden. Es erscheint zweifelhaft, dass in Zeiten knapper Kassen das bisherige
Rechnungswesen den neuen Anforderungen gewachsen ist. Darüber hinaus ist es wenig sinnvoll,
wenn Land und Kommunen in unterschiedlichen Systemen wirtschaften. Wir befürworten daher
eine Reformierung des Rechnungswesens mit der Zielrichtung, den Wechsel von der Kameralistik
zur Doppik in ganz Bayern umzusetzen.
\subsection{Begründung}
\url{http://www.fw-bayern.de/uploads/media/doppik-kameralistik.pdf}

\anderungsantrag
% Antragsnummer
{S05}
% Antragstitel
{Einführung von Mitgliederentscheid}
% Wikiurl zum Antrag
{http://wiki.piratenpartei.de/BY:Landesparteitag 2012.1/Antragsfabrik/Mitgliederentscheid}
% Antragssteller
{Rainer Wutta}
% Betrifft
{Satzung des Landesverbands Bayern / Abschnitt A§14}
% Antragstext
\subsection{Neuer §: Abschnitt A§ 14Mitgliederentscheid}
\begin{enumerate}
	\item Wenn 5 \% der stimmberechtigten Mitglieder zu einem Thema einen Mitgliederentscheid wünschen, wird dieser Entscheid vom jeweiligen Vorstand (Landes-, Bezirks-, Kreis- und Ortsvorstand) innerhalb von 4 Wochen initiiert, durchgeführt und die Ergebnisse veröffentlicht.
	\item Die Durchführung muß geheim per Urnenwahl und per Briefwahl erfolgen.
	\item Das Quorum von 5\% der stimmberechtigten Mitglieder bezieht sich auf die jeweilige Region in der der Mitgliederentscheid durchgeführt wird und für die dieser Entscheid bindend istd.h. Land, Bezirk, Kreis und Gemeinde.
	\item Stimmt die einfache Mehrheit der sich beteiligenden stimmberechtigten Mitglieder des Gebietes
für dieses Thema, so ist es auf der jeweiligen EbeneLand, Bezirk, Kreis und Gemeinde -
angenommen und bindend. Für Satzungsänderungen müssen mehr als 50 \% der stimmberechtigten
Mitglieder des Gebietes für das Thema stimmen.
\end{enumerate}
\subsection{Begründung}
Der Mitgliederentscheid innerhalb der Partei entspricht dem Volksentscheid, den wir Piraten z.B. beim bedingungslosen
Grundeinkommen fordern. Er ist eine Ergänzung zu Parteitagen d.h. die Willensbildung der Mitglieder kann auch zwischen
Parteitagen stattfinden. Dies ist gelebte innerparteiliche Basisdemokratie als Vorbild für ganz Deutschland.

\sonstigerantrag
% Antragsnummer
{X04}
% Antragstitel
{Erstellung des Wahlprogramms}
% Wikiurl zum Antrag
{http://wiki.piratenpartei.de/BY:Landesparteitag 2012.1/Antragsfabrik/Wahlprogramm}
 % Antragssteller
{CEdge}
\konkurrenz{P54}
% Antragstext
\subsection{Antrag}
Der Landesparteitag möge folgendes beschließen:\\
Zur Entwicklung des Wahlprogramms können für den Landesparteitag Positionspapiere und
Programmanträge für das Wahlprogramm eingereicht werden.\par
Der Landesvorstand wird beauftragt, dafür zu sorgen, dass die Inhalte und Forderungen der
Positionspapiere in das Landeswahlprogramm einfließen. Die Texte der Wahlprogrammanträge
werden direkt in das Landeswahlprogramm übernommen.\par
Weiterhin wird der Landesvorstand beauftragt, für eine hinreichende redaktionelle und formelle
(Reihenfolge der Punkte) Ausarbeitung des Landeswahlprogramms zu sorgen. Dies soll zeitig nach
dem Abschluss der inhaltlichen Arbeit stattfinden, sodass das Wahlprogramm spätestens im
Frühjahr 2013 beschlossen werden kann.\par
Falls ein Landesgrundsatzprogramm beschlossen wird, können analog Programmanträge für dieses
gestellt werden. Für diesen Fall wird außerdem der Landesvorstand beauftragt, dafür zu sorgen,
dass dieses redaktionell angemessen aufbereitet wird.
\subsection{Begründung}
Nachdem das im September 2010 vom Landesparteitag beschlossene Konzept zur
Programmentwicklung nur in Teilen umgesetzt wurde, braucht es eine klärende Aussage des
Landesparteitags und einen Beschluss zum weiteren Vorgehen.\par
Einige Piraten haben bereits aufwändig Positionspapiere umgesetzt, in der Erwartung, dass deren
Inhalte wie zuvor beschlossen in das Wahlprogramm einfließen. Es wäre fatal, diese Arbeit zunichte
zu machen. Gleichzeitig haben wir Anträge zum Wahlprogramm vorliegen, die wir nicht einfach
außen vor lassen können und sollten. Also halten wir uns beide Wege offen.\par
So sollen auf den Parteitagen im März und fortfolgend Inhalte beschlossen werden, aus diesen
entsteht das Wahlprogramm. So haben wir ausreichend Zeit und die wesentlichen Inhalte des
Wahlprogramms stehen fest, bevor Listen und Direktkandidaten gewählt werden.\par
Im Antragstext steht bewusst die Formulierung, dass der Landesvorstand dafür Sorge tragen soll,
dass die Inhalte zu einem Landeswahlprogramm aufbereitet werden. Das muss nicht heißen, dass
die Vorstandsmitglieder selbst dies vornehmen müssen, sie tragen aber die Verantwortung dafür,
dass es geschieht.\par
Die Aussage zu einem eventuellen Landesgrundsatzprogramm ist, dass hierfür eigene
Programmanträge gestellt werden sollen und dieses Programm - falls es beschlossen wird -
ebenfalls aufbereitet werden soll, allerdings lediglich redaktionell (Rechtschreibfehler usw.).
\end{document}

